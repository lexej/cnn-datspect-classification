\section{Introduction}
\label{sec:intro}




% Deep Learning advancements in field of medical AI


% Binary classification of medical images -> skin cancer, brain cancer, parkinson


%  Problem domain

There is a variety of quality metrics which can be computed to evaluate the performance of a binary classifier 
given a certain decision boundary or inconclusive range.
In practice, the area under curve (AUC) for a receiver operating characteristic (ROC) curve is a commonly used metric 
for assessing the overall performance of a binary classifier.
Alternatively, the AUC value for the precision-recall curve can be computed as a quality metric. 

For a binary classification problem in the medical domain, it is of particular interest to train both the most confident and 
accurate classifier.
A classifier with high decision confidence applied in clinical practice would require less manual inspection by the
physician which reduces both effort and costs.
Therefore one practically important optimization problem is to minimize the number of cases predicted within the inconclusive range 
by the classifier while maximizing its performance.

% related work? not found such framework

This thesis contributes a model-agnostic and robust evaluation metric for the diagnostic decision confidence of a classifier 
as well as an implementation for producing these evaluation results.
We define the decision confidence to be maximized where the performance of the classifier on consensus test cases 
is maximized over a broad scale of determined inconclusive ranges.
Therefore, as the benchmark method, the specific binding ratio (SBR) of 123I-FP-CIT in the putamen was employed and 
compared to conventional classification methods and convolutional neural network (CNN) approaches.
The different models are trained and tested on subsets of a DAT-SPECT dataset consisting of 1740 slices of 
volumetric DAT-SPECT images.
Data augmentation is applied to the DAT-SPECT dataset to increase the heterogeneity of the training and testing data.
Additionally, the methods are evaluated on the Parkinson's Progression Markers Initiative (PPMI) dataset ZITAT and 
the multiple-pinhole (MPH) dataset ZITAT.

% hypothesis -> CNN based methods should outperform 


% research questions


% thesis structure


