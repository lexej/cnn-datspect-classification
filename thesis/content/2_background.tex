\section{Background}
\label{sec:background}

\subsection{DAT-SPECT for diagnosing PD}
\label{subsec:datspect}

% DAT-SPECT - Medical background

PD, as well as the `atypical' neurodegenerative PS, is associated with progressive loss of substantia nigra pars compacta (SNpc) dopaminergic neurons 
projecting to the striatum [10]. 
Reduced availability of dopamine transporters (DAT) in the striatum is well-validated as a biomarker for nigrostriatal degeneration in PD [11-13]. 
It can be detected by single photon emission computed tomography (SPECT) with dopamine transporter (DAT) ligands [14, 15]. 
Reduction of striatal DAT availability is strongly advanced already at the earliest symptomatic (motor) stages of PD, 
because the degeneration of dopaminergic nerve endings in the striatum is an early step in the pathological PD cascade [11-13]. 
Compensatory downregulation of the DAT expression in the remaining nerve endings results in even more pronounced striatal DAT loss [16-18]. 
Secondary PS are as a rule not associated with nigrostriatal degeneration and loss of striatal DAT. 
To differentiate PD from secondary PS based on striatal DAT availability, the radioactively labeled DAT ligand [$^{123}$I]FP-CIT 
(trade name: $\text{DaTscan}^{\copyright}$) has been licensed as SPECT tracer in both, the US and Europe [19].

% DAT-SPECT 

A recent review, including a non-systematic meta-analysis, of DAT-SPECT with [$^{123}$I]FP-CIT in PS confirmed high sensitivity (median 93\%) 
and high specificity (median 89\%) of DAT-SPECT for the differentiation of PD from secondary PS in patients with CUPS [20]. 
The review further revealed that DAT-SPECT leads to a change of diagnosis in about 40\% and to a change of treatment in about the same proportion of 
patients with CUPS [20]. 
Thus, DAT-SPECT with [$^{123}$I]FP-CIT is highly diagnostically accurate and has a relevant impact on the diagnosis and treatment of CUPS patients. 
Guidelines from professional neurological societies therefore strongly strengthened the role of DAT-SPECT with [$^{123}$I]FP-CIT in the last years [21]. 
For example, the current version of the S3 guideline “Idiopathic Parkinson syndrome” of the German Society of Neurology states that DAT-SPECT 
\textit{should} be performed at an early disease stage in CUPS. 


\subsection{Methods for classification}
\label{subsec:randfors}

% CNN

% Random forest 

