\section{Background}
\label{sec:background}

\subsection{DAT-SPECT for Detecting Parkinson's Disease}
\label{subsec:datspect}

% DAT-SPECT - Medical background

Parkinson's disease (PD) and 'atypical' neurodegenerative Parkinsonian syndromes (PS) 
are both associated with the progressive loss of dopaminergic neurons in the substantia nigra pars compacta (SNpc) 
that project to the striatum~\citep{Piggott1999}.
The reduced availability of dopamine transporters (DAT) in the striatum is a well-validated biomarker 
for nigrostriatal degeneration in PD~\citep{Bernheimer1973, Fazio2018, Niznik1991}.
It can be detected by single photon emission computed tomography (SPECT) 
with dopamine transporter (DAT) ligands~\citep{Kuikka1995, Abi-Dargham1996}.
The reduction in striatal DAT availability is significantly advanced 
even in the earliest symptomatic (motor) stages of PD, 
as the degeneration of dopaminergic nerve endings in the striatum represents 
an early step in the pathological PD cascade~\citep{Bernheimer1973, Fazio2018, Niznik1991}.
The compensatory downregulation of the DAT expression in the remaining nerve endings 
leads to a more pronounced loss of striatal DAT~\citep{Lee2000, Saari2017, Honkanen2019}.
Secondary PS's are typically not associated with nigrostriatal degeneration or the loss of striatal DAT. 
To differentiate PD from secondary PS based on striatal DAT availability, 
the radiolabeled DAT ligand [$^{123}$I]FP-CIT (trade name: $\text{DaTscan}^{\copyright}$) 
has been approved as a SPECT tracer in both the US and Europe~\citep{Neumeyer1994}.

% DAT-SPECT for differentiation of PD from PS's

A recent review, which involved a non-systematic meta-analysis of DAT-SPECT with [$^{123}$I]FP-CIT in patients with PS, 
confirmed that DAT-SPECT exhibits high sensitivity (median 93\%) and high specificity (median 89\%) 
in differentiating PD from secondary PS in patients with 
clinically uncertain parkinsonian syndrome (CUPS)~\citep{Buchert2019-ya}.
Moreover, the review demonstrated that DAT-SPECT results in a change in diagnosis for about 40\% of patients with CUPS
and leads to a change in treatment for a similar proportion of these patients~\citep{Buchert2019-ya}. 
Thus, DAT-SPECT with [$^{123}$I]FP-CIT is a highly accurate diagnostic method
that significantly influences the diagnosis and treatment of patients with CUPS.
Guidelines from professional neurological societies have therefore strongly emphasized 
the role of DAT-SPECT with [$^{123}$I]FP-CIT in recent years~\citep{Tatsch2013}.
For example, the current version of the S3 guideline “Idiopathic Parkinson syndrome” of the 
German Society of Neurology states that DAT-SPECT \textit{should} be conducted at an early disease stage in CUPS patients.

\subsection{Convolutional Neural Networks for Image Classification}
\label{subsec:randfors}

% CNN

% Random forest 

