\section{Data Sources}
\label{sec:data}

The study retrospectively included 3 different datasets with a total of 3025 DAT-SPECT images.
The primary dataset (``development dataset") was used for both training and testing the models associated with the respective method, 
whereas the other two datasets, the \textit{PPMI} dataset and the \textit{MPH} dataset were used for testing only, not for training.

\subsection{Development dataset}
\label{subsec:spect_dataset}

The development dataset comprised 1740 consecutive DAT-SPECT from clinical routine at our site as described in [56]. 
In brief, DAT-SPECT with [$^{123}$I]FP-CIT had been performed according to common procedures guidelines [57, 58] with different double-head cameras 
equipped with low-energy-high-resolution or fan-beam collimators. 
The projection data were reconstructed using the iterative ordered-subsets-expectation-maximization [59] with attenuation and simulation-based scatter correction 
as well as collimator-detector response modeling as implemented in the Hybrid Recon-Neurology tool of the Hermes SMART workstation v1.6 (Hermes Medical Solutions, Stockholm, Sweden) [60-63]. 
All parameter settings were as recommended by Hermes [60] for the EANM / EANM Research Ltd (EARL) ENC-DAT project (European Normal Control Database of DaTSCAN) [64-68]. 
More precisely, ordered-subsets-expectation-maximization was performed with 5 iterations and 15/16 subsets for 120/128 views. 
For noise suppression, reconstructed images were postfiltered by convolution with a 3-dimensional Gaussian kernel of 7 mm full-width-at-half-maximum. 
The development dataset was used for both, training and testing. For this purpose, the dataset was randomly split into ??? training cases and ??? test cases. 
The gold standard label as either “normal” or Parkinson-typical reduction (“reduced”) of the striatal signal had been obtained by visual interpretation of the DAT-SPECT images by 3 independent readers [56]. 
The between-reader consensus on the label could not be achieved for around 5\% of dataset cases.

\subsection{Independent testing datasets}
\label{subsec:external_dataset}

The second dataset comprised 645 DAT-SPECT with [$^{123}$I]FP-CIT from the Parkinson's Progression Markers Initiative (PPMI) (www.ppmi-info.org/data) [69]. 
The external dataset included 438 patients with Parkinson's disease and 207 healthy controls as described in [46]. 
Details of the PPMI DAT-SPECT protocol are given at http://www.ppmi-info.org/study-design/research-documents-and-sops/ [69]. 
Raw projection data has been transferred to the PPMI imaging core lab for central image reconstruction using an iterative (HOSEM) algorithm on a HERMES workstation. 
The clinical diagnosis was used as gold standard label (Parkinson's disease = ``reduced", healthy control = ``normal"). 
The external dataset showed lower spatial resolution than the development dataset (lower striatum-to-background contrast).

The third dataset (``MPH dataset") comprised 640 consecutive DAT-SPECT with [$^{123}$I]FP-CIT from clinical routine at UKE that had been acquired with a triple-head camera equipped with brain-specific multiple pinhole collimators. 
Multiple pinhole SPECT concurrently improves count sensitivity and spatial resolution compared to SPECT with parallel-hole and fan-beam collimators [70, 71]. 
The projection data were reconstructed with the Monte Carlo photon simulation engine and iterative one-step-late maximum-a-posteriori expectation-maximization implemented in the camera software (24 iterations, 2 subsets) [71, 72]. 
Neither attenuation nor scatter correction was applied. 
The gold standard label (``normal" or ``reduced") was obtained by visual interpretation by an experienced reader 
(about 20 years of experience in clinical DAT-SPECT reading, $\geq$3,000 cases).
All SPECT images were interpreted twice (with different randomization) by the same reader. 
The delay between the reading sessions was 14 days. 
Cases with discrepant interpretations between the two reading sessions were read a third time by the same reader to obtain an intra-reader consensus as the gold standard label. 
The MPH test dataset has not been described previously.
Compared to the development dataset, the internal test dataset was characterized by better spatial resolution 
(resulting in higher striatum-to-background contrast) and less statistical noise. 
