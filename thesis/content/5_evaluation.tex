\section{Evaluation}
\label{sec:evaluation}

The preceding chapters have detailed the research methodology, data collection and sources, and the application of classification techniques 
to address the research questions posed in this study. 

This chapter embarks on the evaluation of the research results, focusing on the performance and effectiveness of the methods employed, 
and the attainment of the research objectives.

The structure of this chapter has been designed to systematically lead readers through the assessment process. 
It commences with a discussion of the research objectives that serve as the focal points for subsequent evaluations. 
Following this, a comprehensive examination of performance metrics is conducted, with an emphasis on their 
significance within the context of this research, providing detailed insights into the criteria employed 
to evaluate research outcomes.
The core of this chapter subsequently unveils the experimental results encompassing various test datasets 
and classification methods. 
These findings are presented using graphical representations and supported by a range of statistical measures.
The chapter culminates with a comparative analysis, which seeks to assess and contrast the effectiveness and 
limitations of the research methods employed.

\subsection{Research Objectives}
\label{subsec:research_objectives}

% TODO

% Objective 1: Evaluation of Methods on Test set of dev data

%   Objective 1.1: To compare the "performance" of the methods on Test set of dev data.
%   Objective 1.2: To assess the suitability of these methods in unseen real-world scenarios


% Objective 2: Evaluation of Methods on PPMI and MPH dataset
%				-> understand the impact of varying dataset characteristics by evaluating on


\subsection{Evaluation Metrics and Procedure}
\label{subsec:determinationInconcl}

In the following the performance metrics used for the evaluation of the different classification methods 
are explained in more detail.

First the $\text{mean} \pm \text{SD}$ (standard deviation) of the following measures were calculated across
the different random splits for each classification approach given a cutoff: 
Area Under Curve (AUC) for ROC curve, Balanced accuracy, accuracy, sensitivity, specificity, 
Positive Predictive Value (PPV) and Negative Predictive Value (NPV).
The natural cutoff of 0.5 was used for each classification approach except the SBR method.
For the SBR method the optimal cutoff was determined using the Youden criterion~\citep{Youden1950} and was used for
calculating the measures.

% Determination of inconclusive intervals given set of percentages of inconclusive cases
Second for each element within a set of considered percentages of inconclusive cases in the validation set 
the corresponding inconclusive interval was determined.
Inconclusive cases were defined as cases predicted within an inconclusive interval 
(bounded by lower and upper bound), while conclusive cases were those predicted outside this interval.
The determination of the inconclusive interval was exclusively performed using the validation set 
for each random split and classification approach independently.
The set of percentages of inconclusive cases considered ranged from 0.2\% to 20.0\%, increasing in increments of 0.2\%.
For each target percentage of inconclusive cases in the set the lower and upper bounds of the inconclusive interval 
were independently determined in such a way that there was a similar number of inconclusive cases both below and above 
the pre-defined cutoff.
For the CNN-based classification approaches (described in Section~\ref{subsec:cnn_based_classification}) and the 
multivariate benchmark (described in Section~\ref{subsec:pca_rfc}) the natural cutoff of 0.5 was used.
For the SBR-based univariate benchmark (described in Section~\ref{subsec:sbr}), 
the optimal cutoff on the SBR obtained by applying the Youden criterion~\citep{Youden1950} using ROC analysis was used.

% Stability of inconclusive interval
To assess the stability of the determined inconclusive interval over the proportion of inconclusive cases
the determined upper and lower bounds ($\text{mean} \pm \text{SD}$) of the inconclusive interval
were plotted against the corresponding proportion of inconclusive cases (\%) in the validation set.
The $\text{mean} \pm \text{SD}$ of determined upper and lower bounds was calculated across the measures for 
different random splits.
The rate at which the lower (upper) bound decreases (increases) in relation to the proportion of inconclusive 
cases reflects the density of inconclusive cases around the cutoff.
Specifically, higher function gradients indicate lower concentration of predictions around the cutoff, 
and vice versa.
The measurement was conducted separately for each classification approach.

% AUC of bal_acc vs. %inconclusive_cases; determination of inconclusive ranges for each %inconclusive_cases

The main performance metric used in this work to evaluate and compare the classification approaches was 
the area under the curve (AUC) of mean balanced accuracy (\%) on conclusive test cases as a function of 
the proportion of inconclusive test cases (\%, mean).
More precisely the relative AUC (\%) normalized to the maximum achievable area was used for the comparison.
To obtain the relative AUC, 
first, the mean balanced accuracy function was interpolated using cubic spline interpolation.
Then the area under the mean balanced accuracy curve was computed using the trapezoidal rule 
and then normalized to the maximum achievable area.
The evaluation of each classification method with respect to this metric was conducted on the test set of the 
development dataset as well as on the independent datasets PPMI and MPH.


% Proportion of inconclusive cases in test set vs. 
Also the observed $\text{mean} \pm \text{SD}$ proportion of inconclusive cases in the test set was plotted 
against the proportion of inconclusive cases in the validation set.
% WHY? Explain what would indicate desired behaviour and what not


%%%%%%%%%%%%%%%%%%%%%%%%%%%%%%%%%%%%%%%%%%%%%%%%%%%%%%%%%%%%%%%%%%%%%%%%%%%%%%%%%%%%%%%%%%%%%%%%%%%%%%%%%%%%%%%%%%%%%%%

\subsection{Evaluation of SBR Method}
\label{subsec:eval_sbr}

%%%%%%%%%%%%%%%%%%%%%%%%%%%%%%%%%%%%%%%%%%%%%%%%%%%%%%%%%%%%%%%%%%%%%%%%%%%%%%%%%%%%%%%%%%%%%%%%%%%%%%%%%%%%%%%%%%%%%%%

% Evaluation on Development dataset


\begin{table}[ht]
  \caption{Evaluation of the SBR method on Development dataset. 
  For evaluation, the optimal SBR cutoff 0.703$\pm$0.009 was used (variation across random splits).}
  \centering
  \begin{tabular}{llll}
      \hline
                        & train set         & validation set      & test set             \\
      \hline
      Balanced Accuracy & 0.936$\pm$0.003   &   0.929$\pm$0.008   &  0.935$\pm$0.007     \\
      Accuracy          & 0.936$\pm$0.003   &   0.930$\pm$0.008   &  0.935$\pm$0.007     \\
      Sensitivity       &  0.934$\pm$0.006  &   0.924$\pm$0.005   &  0.930$\pm$0.014     \\
      Specificity       & 0.937$\pm$0.003   &   0.935$\pm$0.015   &  0.939$\pm$0.012     \\
      PPV               &  0.933$\pm$0.005  &   0.929$\pm$0.014   &  0.930$\pm$0.015     \\
      NPV               &  0.938$\pm$0.005  &   0.930$\pm$0.004   &  0.938$\pm$0.018     \\
      \hline
      AUC-ROC          &  \multicolumn{3}{c}{0.983$\pm$0.002 }  \\
      \hline
  \end{tabular}
 \label{t1:sbr_perf_eval_table}
\end{table}

% sbr_percInconclCases_development
\begin{figure}[t]
    \centering
    \includegraphics[width=1.0\textwidth]{content/figures/evaluations/sbr/86/sbr_percInconclCases_development.png}
    \caption{Evaluation of the SBR method on Test Set of Development Dataset. 
    Determined upper and lower bounds of the inconclusive interval as a function of the percentage of inconclusive cases.} 
    \label{fig:sbr_percInconclCases_development}
\end{figure}


% obsInconclCases_inconclCasesValid_sbr_development
\begin{figure}[h]
    \centering
    \includegraphics[width=1.0\textwidth]{content/figures/evaluations/sbr/86/obsInconclCases_inconclCasesValid_sbr_development.png}
    \caption{Evaluation of the SBR method on Test Set of Development Dataset.
    Observed percentage of inconclusive cases in the test set 
    for a given set of percentages of inconclusive cases in the validation set.
    Each of the percentages of inconclusive cases in the validation set is associated 
    with an inconclusive range (determined in the validation set).} 
    \label{fig:obsInconclCases_inconclCasesValid_sbr_development}
\end{figure} 


% bacc_obsInconclCases_sbr_development_full
\begin{figure}[t]
    \begin{subfigure}{0.9\textwidth}
      \centering
      \includegraphics[width=0.9\textwidth]{content/figures/evaluations/sbr/86/bacc_obsInconclCases_sbr_development.png}
      \subcaption{}
      \label{fig:bacc_obsInconclCases_sbr_development}
    \end{subfigure}
    \hfill
    \begin{subfigure}{0.9\textwidth}
      \centering
      \includegraphics[width=0.9\textwidth]{content/figures/evaluations/sbr/86/bacc_obsInconclCases_concl_sbr_development.png}
      \subcaption{}
      \label{fig:bacc_obsInconclCases_concl_sbr_development}
    \end{subfigure}

    \caption{Evaluation of the SBR method on Test Set of Development Dataset.
    Balanced accuracy for a given mean percentage of observed inconclusive cases in the test set on 
    (a) both conclusive and inconclusive cases and (b) only conclusive cases. 
    Each of the mean percentages of observed inconclusive cases is associated with an inconclusive range (determined in the validation set). }
    \label{fig:bacc_obsInconclCases_sbr_development_full}
\end{figure}

%%%%%%%%%%%%%%%%%%%%%%%%%%%%%%%%%%%%%%%%%%%%%%%%%%%%%%%%%%%%%%%%%%%%%%%%%%%%%%%%%%%%%%%%%%%%%%%%%%%%%%%%%%%%%%%%%%%%%%%

% Evaluation on Independent datasets

% ---------------- PPMI ---------------------

% obsInconclCases_inconclCasesValid_sbr_ppmi and bacc_obsInconclCases_concl_sbr_ppmi
\begin{figure}[t]
  \begin{subfigure}{0.9\textwidth}
    \centering
    \includegraphics[width=0.9\textwidth]{content/figures/evaluations/sbr/86/obsInconclCases_inconclCasesValid_sbr_ppmi.png}
    \subcaption{Observed percentage of inconclusive cases in the PPMI dataset 
    for a given set of percentages of inconclusive cases in the validation set (Development dataset).} 
    \label{fig:obsInconclCases_inconclCasesValid_sbr_ppmi}
  \end{subfigure}
  \hfill
  \begin{subfigure}{0.9\textwidth}
    \centering
    \includegraphics[width=0.9\textwidth]{content/figures/evaluations/sbr/86/bacc_obsInconclCases_concl_sbr_ppmi.png}
    \subcaption{Balanced accuracy on conclusive cases for a given mean percentage of inconclusive cases observed 
    in the PPMI dataset.}
    \label{fig:bacc_obsInconclCases_concl_sbr_ppmi}
  \end{subfigure}
  \caption{Evaluation of the SBR method on PPMI dataset.}
\end{figure}



% -------------- MPH -----------------


% obsInconclCases_inconclCasesValid_sbr_mph and bacc_obsInconclCases_concl_sbr_mph
\begin{figure}[t]
  \begin{subfigure}{0.9\textwidth}
    \centering
    \includegraphics[width=0.9\textwidth]{content/figures/evaluations/sbr/86/obsInconclCases_inconclCasesValid_sbr_mph.png}
    \subcaption{Observed percentage of inconclusive cases in the MPH dataset 
    for a given set of percentages of inconclusive cases in the validation set (Development dataset).} 
    \label{fig:obsInconclCases_inconclCasesValid_sbr_mph}
  \end{subfigure}
  \hfill
  \begin{subfigure}{0.9\textwidth}
    \centering
    \includegraphics[width=0.9\textwidth]{content/figures/evaluations/sbr/86/bacc_obsInconclCases_concl_sbr_mph.png}
    \subcaption{Balanced accuracy on conclusive cases for a given mean percentage of inconclusive cases observed 
    in the MPH dataset.}
    \label{fig:bacc_obsInconclCases_concl_sbr_mph}
  \end{subfigure}
  \caption{Evaluation of the SBR method on MPH dataset.}
\end{figure}



%%%%%%%%%%%%%%%%%%%%%%%%%%%%%%%%%%%%%%%%%%%%%%%%%%%%%%%%%%%%%%%%%%%%%%%%%%%%%%%%%%%%%%%%%%%%%%%%%%%%%%%%%%%%%%%%%%%%%%%

\subsection{Evaluation of Random Forest Method}
\label{subsec:eval_rfc}


%%%%%%%%%%%%%%%%%%%%%%%%%%%%%%%%%%%%%%%%%%%%%%%%%%%%%%%%%%%%%%%%%%%%%%%%%%%%%%%%%%%%%%%%%%%%%%%%%%%%%%%%%%%%%%%%%%%%%%%

% Evaluation on Development dataset

\begin{table}[ht]
  \caption{Evaluation of the PCA-RFC method on Development dataset. 
  For evaluation, the natural sigmoid cutoff $0.5$ was used.}
  \centering
  \begin{tabular}{llll}
      \hline
                        & train set         & validation set      & test set             \\
      \hline
      Balanced Accuracy & 1.000$\pm$0.000   &   0.963$\pm$0.010    &  0.966$\pm$0.006   \\
      Accuracy          & 1.000$\pm$0.000    &   0.963$\pm$0.010  &  0.966$\pm$0.006    \\
      Sensitivity       &  1.000$\pm$0.000   &   0.957$\pm$0.012   &  0.962$\pm$0.010   \\
      Specificity       & 1.000$\pm$0.000    &   0.969$\pm$0.011  &  0.969$\pm$0.009   \\
      PPV               &  1.000$\pm$0.000   &   0.966$\pm$0.012   &  0.965$\pm$0.010   \\
      NPV               &  1.000$\pm$0.000   &   0.961$\pm$0.012  &  0.966$\pm$0.011   \\
      \hline
      AUC-ROC          &  \multicolumn{3}{c}{0.994$\pm$0.002}  \\
      \hline
  \end{tabular}
 \label{t1:erc_perf_eval_table}
\end{table}

% pca_rfc_percInconclCases_development
\begin{figure}[t]
  \centering
  \includegraphics[width=1.0\textwidth]{content/figures/evaluations/pca_rfc/86/sigmoid_percInconclCases_pca_rfc_development.png}
  \caption{Evaluation of the PCA-RFC method on Test Set of Development Dataset. 
  Determined upper and lower bounds of the inconclusive interval as a function of the percentage of inconclusive cases.} 
  \label{fig:pca_rfc_percInconclCases_development}
\end{figure}


% obsInconclCases_inconclCasesValid_pca_rfc_development
\begin{figure}[h]
  \centering
  \includegraphics[width=1.0\textwidth]{content/figures/evaluations/pca_rfc/86/obsInconclCases_inconclCasesValid_pca_rfc_development.png}
  \caption{Evaluation of the PCA-RFC method on Test Set of Development Dataset.
  Observed percentage of inconclusive cases in the test set 
  for a given set of percentages of inconclusive cases in the validation set.
  Each of the percentages of inconclusive cases in the validation set is associated 
  with an inconclusive range (determined in the validation set).} 
  \label{fig:obsInconclCases_inconclCasesValid_pca_rfc_development}
\end{figure} 


% bacc_obsInconclCases_pca_rfc_development_full
\begin{figure}[t]
  \begin{subfigure}{0.9\textwidth}
    \centering
    \includegraphics[width=0.9\textwidth]{content/figures/evaluations/pca_rfc/86/bacc_obsInconclCases_pca_rfc_development.png}
    \subcaption{}
    \label{fig:bacc_obsInconclCases_pca_rfc_development}
  \end{subfigure}
  \hfill
  \begin{subfigure}{0.9\textwidth}
    \centering
    \includegraphics[width=0.9\textwidth]{content/figures/evaluations/pca_rfc/86/bacc_obsInconclCases_concl_pca_rfc_development.png}
    \subcaption{}
    \label{fig:bacc_obsInconclCases_concl_pca_rfc_development}
  \end{subfigure}

  \caption{Evaluation of the PCA-RFC method on Test Set of Development Dataset.
  Balanced accuracy for a given mean percentage of observed inconclusive cases in the test set on 
  (a) both conclusive and inconclusive cases and (b) only conclusive cases. 
  Each of the mean percentages of observed inconclusive cases is associated with an inconclusive range (determined in the validation set). }
  \label{fig:bacc_obsInconclCases_pca_rfc_development_full}
\end{figure}

%%%%%%%%%%%%%%%%%%%%%%%%%%%%%%%%%%%%%%%%%%%%%%%%%%%%%%%%%%%%%%%%%%%%%%%%%%%%%%%%%%%%%%%%%%%%%%%%%%%%%%%%%%%%%%%%%%%%%%%

% Evaluation on Independent datasets

% ---------------- PPMI ---------------------

% obsInconclCases_inconclCasesValid_pca_rfc_ppmi and bacc_obsInconclCases_concl_pca_rfc_ppmi
\begin{figure}[t]
  \begin{subfigure}{0.9\textwidth}
    \centering
    \includegraphics[width=0.9\textwidth]{content/figures/evaluations/pca_rfc/86/obsInconclCases_inconclCasesValid_pca_rfc_ppmi.png}
    \subcaption{Observed percentage of inconclusive cases in the PPMI dataset 
    for a given set of percentages of inconclusive cases in the validation set (Development dataset).} 
    \label{fig:obsInconclCases_inconclCasesValid_pca_rfc_ppmi}
  \end{subfigure}
  \hfill
  \begin{subfigure}{0.9\textwidth}
    \centering
    \includegraphics[width=0.9\textwidth]{content/figures/evaluations/pca_rfc/86/bacc_obsInconclCases_concl_pca_rfc_ppmi.png}
    \subcaption{Balanced accuracy on conclusive cases for a given mean percentage of inconclusive cases observed 
    in the PPMI dataset.}
    \label{fig:bacc_obsInconclCases_concl_pca_rfc_ppmi}
  \end{subfigure}
  \caption{Evaluation of the PCA-RFC method on PPMI dataset.}
\end{figure}



% -------------- MPH -----------------


% obsInconclCases_inconclCasesValid_pca_rfc_mph and bacc_obsInconclCases_concl_pca_rfc_mph
\begin{figure}[t]
  \begin{subfigure}{0.9\textwidth}
    \centering
    \includegraphics[width=0.9\textwidth]{content/figures/evaluations/pca_rfc/86/obsInconclCases_inconclCasesValid_pca_rfc_mph.png}
    \subcaption{Observed percentage of inconclusive cases in the MPH dataset 
    for a given set of percentages of inconclusive cases in the validation set (Development dataset).} 
    \label{fig:obsInconclCases_inconclCasesValid_pca_rfc_mph}
  \end{subfigure}
  \hfill
  \begin{subfigure}{0.9\textwidth}
    \centering
    \includegraphics[width=0.9\textwidth]{content/figures/evaluations/pca_rfc/86/bacc_obsInconclCases_concl_pca_rfc_mph.png}
    \subcaption{Balanced accuracy on conclusive cases for a given mean percentage of inconclusive cases observed 
    in the MPH dataset.}
    \label{fig:bacc_obsInconclCases_concl_pca_rfc_mph}
  \end{subfigure}
  \caption{Evaluation of the PCA-RFC method on MPH dataset.}
\end{figure}



\subsection{Evaluation of MVT-based Method}
\label{subsec:eval_mvt}



%%%%%%%%%%%%%%%%%%%%%%%%%%%%%%%%%%%%%%%%%%%%%%%%%%%%%%%%%%%%%%%%%%%%%%%%%%%%%%%%%%%%%%%%%%%%%%%%%%%%%%%%%%%%%%%%%%%%%%%

% Evaluation on Development dataset


\begin{table}[ht]
  \caption{Evaluation of the CNN-MVT method on Development dataset. 
  For evaluation, the natural sigmoid cutoff $0.5$ was used.}
  \centering
  \begin{tabular}{llll}
      \hline
                        & train set         & validation set      & test set             \\
      \hline
      Balanced Accuracy & 0.999$\pm$0.003   &  0.970$\pm$0.014    &  0.970$\pm$0.008   \\
      Accuracy          & 0.999$\pm$0.003    &   0.970$\pm$0.014   &  0.970$\pm$0.008   \\
      Sensitivity       &  1.000$\pm$0.000   &   0.963$\pm$0.010   &  0.964$\pm$0.015  \\
      Specificity       &   0.997$\pm$0.006   &   0.976$\pm$0.023  &   0.976$\pm$0.013  \\
      PPV               &  0.997$\pm$0.006   &   0.975$\pm$0.024   &  0.972$\pm$0.018 \\
      NPV               &  1.000$\pm$0.000    &   0.966$\pm$0.010   & 0.968$\pm$0.014  \\
      \hline
      AUC-ROC          &  \multicolumn{3}{c}{0.996$\pm$0.002}  \\
      \hline
  \end{tabular}
 \label{t1:cnn_mvt_perf_eval_table}
\end{table}


% baseline_majority_percInconclCases_development
\begin{figure}[t]
  \centering
  \includegraphics[width=1.0\textwidth]{content/figures/evaluations/baseline_majority/86/sigmoid_percInconclCases_baseline_majority_development.png}
  \caption{Evaluation of the CNN-MVT method on Test Set of Development Dataset. 
  Determined upper and lower bounds of the inconclusive interval as a function of the percentage of inconclusive cases.} 
  \label{fig:baseline_majority_percInconclCases_development}
\end{figure}


% obsInconclCases_inconclCasesValid_baseline_majority_development
\begin{figure}[h]
  \centering
  \includegraphics[width=1.0\textwidth]{content/figures/evaluations/baseline_majority/86/obsInconclCases_inconclCasesValid_baseline_majority_development.png}
  \caption{Evaluation of the CNN-MVT method on Test Set of Development Dataset.
  Observed percentage of inconclusive cases in the test set 
  for a given set of percentages of inconclusive cases in the validation set.
  Each of the percentages of inconclusive cases in the validation set is associated 
  with an inconclusive range (determined in the validation set).} 
  \label{fig:obsInconclCases_inconclCasesValid_baseline_majority_development}
\end{figure} 


% bacc_obsInconclCases_baseline_majority_development_full
\begin{figure}[t]
  \begin{subfigure}{0.9\textwidth}
    \centering
    \includegraphics[width=0.9\textwidth]{content/figures/evaluations/baseline_majority/86/bacc_obsInconclCases_baseline_majority_development.png}
    \subcaption{}
    \label{fig:bacc_obsInconclCases_baseline_majority_development}
  \end{subfigure}
  \hfill
  \begin{subfigure}{0.9\textwidth}
    \centering
    \includegraphics[width=0.9\textwidth]{content/figures/evaluations/baseline_majority/86/bacc_obsInconclCases_concl_baseline_majority_development.png}
    \subcaption{}
    \label{fig:bacc_obsInconclCases_concl_baseline_majority_development}
  \end{subfigure}

  \caption{Evaluation of the CNN-MVT method on Test Set of Development Dataset.
  Balanced accuracy for a given mean percentage of observed inconclusive cases in the test set on 
  (a) both conclusive and inconclusive cases and (b) only conclusive cases. 
  Each of the mean percentages of observed inconclusive cases is associated with an inconclusive range (determined in the validation set). }
  \label{fig:bacc_obsInconclCases_baseline_majority_development_full}
\end{figure}

%%%%%%%%%%%%%%%%%%%%%%%%%%%%%%%%%%%%%%%%%%%%%%%%%%%%%%%%%%%%%%%%%%%%%%%%%%%%%%%%%%%%%%%%%%%%%%%%%%%%%%%%%%%%%%%%%%%%%%%

% Evaluation on Independent datasets

% ---------------- PPMI ---------------------

% obsInconclCases_inconclCasesValid_baseline_majority_ppmi
\begin{figure}[h]
\centering
\includegraphics[width=1.0\textwidth]{content/figures/evaluations/baseline_majority/86/obsInconclCases_inconclCasesValid_baseline_majority_ppmi.png}
\caption{Evaluation of the CNN-MVT method on PPMI Dataset.
Observed percentage of inconclusive cases in the PPMI dataset 
for a given set of percentages of inconclusive cases in the validation set (Development Dataset).
Each of the percentages of inconclusive cases in the validation set is associated 
with an inconclusive range (determined in the validation set).} 
\label{fig:obsInconclCases_inconclCasesValid_baseline_majority_ppmi}
\end{figure} 


% bacc_obsInconclCases_baseline_majority_ppmi_full
\begin{figure}[t]
\begin{subfigure}{0.9\textwidth}
  \centering
  \includegraphics[width=0.9\textwidth]{content/figures/evaluations/baseline_majority/86/bacc_obsInconclCases_baseline_majority_ppmi.png}
  \subcaption{}
  \label{fig:bacc_obsInconclCases_baseline_majority_ppmi}
\end{subfigure}
\hfill
\begin{subfigure}{0.9\textwidth}
  \centering
  \includegraphics[width=0.9\textwidth]{content/figures/evaluations/baseline_majority/86/bacc_obsInconclCases_concl_baseline_majority_ppmi.png}
  \subcaption{}
  \label{fig:bacc_obsInconclCases_concl_baseline_majority_ppmi}
\end{subfigure}

\caption{Evaluation of the CNN-MVT method on PPMI Dataset.
Balanced accuracy for a given mean percentage of inconclusive cases observed in the PPMI dataset on 
(a) both conclusive and inconclusive cases and (b) only conclusive cases. 
Each of the mean percentages of observed inconclusive cases is associated 
with an inconclusive range (determined in the validation set). }
\label{fig:bacc_obsInconclCases_baseline_majority_ppmi_full}
\end{figure}



% -------------- MPH -----------------


% obsInconclCases_inconclCasesValid_baseline_majority_mph
\begin{figure}[h]
\centering
\includegraphics[width=1.0\textwidth]{content/figures/evaluations/baseline_majority/86/obsInconclCases_inconclCasesValid_baseline_majority_mph.png}
\caption{Evaluation of the CNN-MVT method on MPH Dataset.
Observed percentage of inconclusive cases in the MPH dataset 
for a given set of percentages of inconclusive cases in the validation set (Development Dataset).
Each of the percentages of inconclusive cases in the validation set is associated 
with an inconclusive range (determined in the validation set).} 
\label{fig:obsInconclCases_inconclCasesValid_baseline_majority_mph}
\end{figure} 


% bacc_obsInconclCases_baseline_majority_mph_full
\begin{figure}[t]
\begin{subfigure}{0.9\textwidth}
  \centering
  \includegraphics[width=0.9\textwidth]{content/figures/evaluations/baseline_majority/86/bacc_obsInconclCases_baseline_majority_mph.png}
  \subcaption{}
  \label{fig:bacc_obsInconclCases_baseline_majority_mph}
\end{subfigure}
\hfill
\begin{subfigure}{0.9\textwidth}
  \centering
  \includegraphics[width=0.9\textwidth]{content/figures/evaluations/baseline_majority/86/bacc_obsInconclCases_concl_baseline_majority_mph.png}
  \subcaption{}
  \label{fig:bacc_obsInconclCases_concl_baseline_majority_mph}
\end{subfigure}

\caption{Evaluation of the CNN-MVT method on MPH Dataset.
Balanced accuracy for a given mean percentage of inconclusive cases observed in the MPH dataset on 
(a) both conclusive and inconclusive cases and (b) only conclusive cases. 
Each of the mean percentages of observed inconclusive cases is associated 
with an inconclusive range (determined in the validation set). }
\label{fig:bacc_obsInconclCases_baseline_majority_mph_full}
\end{figure}


\subsection{Evaluation of RLT-based Method}
\label{subsec:eval_rlt}


%%%%%%%%%%%%%%%%%%%%%%%%%%%%%%%%%%%%%%%%%%%%%%%%%%%%%%%%%%%%%%%%%%%%%%%%%%%%%%%%%%%%%%%%%%%%%%%%%%%%%%%%%%%%%%%%%%%%%%%

% Evaluation on Development dataset


\begin{table}[ht]
  \caption{Evaluation of the CNN-RLT method on Development dataset. 
  For evaluation, the natural sigmoid cutoff $0.5$ was used.}
  \centering
  \begin{tabular}{llll}
      \hline
                        & train set         & validation set      & test set             \\
      \hline
      Balanced Accuracy & 0.982$\pm$0.003   &  0.967$\pm$0.008    &  0.973$\pm$0.005  \\
      Accuracy          & 0.982$\pm$0.003    &   0.968$\pm$0.008   &  0.973$\pm$0.005  \\
      Sensitivity       &  0.980$\pm$0.008  &   0.951$\pm$0.013  &  0.961$\pm$0.014 \\
      Specificity       &   0.983$\pm$0.008   &   0.984$\pm$0.005  &   0.985$\pm$0.010 \\
      PPV               &  0.983$\pm$0.009   &   0.982$\pm$0.006   &  0.982$\pm$0.012   \\
      NPV               &  0.981$\pm$0.008   &   0.956$\pm$0.012   & 0.966$\pm$0.013  \\
      \hline
      AUC-ROC          &  \multicolumn{3}{c}{0.994$\pm$0.002}  \\
      \hline
  \end{tabular}
 \label{t1:cnn_rlt_perf_eval_table}
\end{table}


% baseline_random_percInconclCases_development
\begin{figure}[t]
  \centering
  \includegraphics[width=1.0\textwidth]{content/figures/evaluations/baseline_random/86/sigmoid_percInconclCases_baseline_random_development.png}
  \caption{Evaluation of the CNN-RLT method on Test Set of Development Dataset. 
  Determined upper and lower bounds of the inconclusive interval as a function of the percentage of inconclusive cases.} 
  \label{fig:baseline_random_percInconclCases_development}
\end{figure}


% obsInconclCases_inconclCasesValid_baseline_random_development
\begin{figure}[h]
  \centering
  \includegraphics[width=1.0\textwidth]{content/figures/evaluations/baseline_random/86/obsInconclCases_inconclCasesValid_baseline_random_development.png}
  \caption{Evaluation of the CNN-RLT method on Test Set of Development Dataset.
  Observed percentage of inconclusive cases in the test set 
  for a given set of percentages of inconclusive cases in the validation set.
  Each of the percentages of inconclusive cases in the validation set is associated 
  with an inconclusive range (determined in the validation set).} 
  \label{fig:obsInconclCases_inconclCasesValid_baseline_random_development}
\end{figure} 


% bacc_obsInconclCases_baseline_random_development_full
\begin{figure}[t]
  \begin{subfigure}{0.9\textwidth}
    \centering
    \includegraphics[width=0.9\textwidth]{content/figures/evaluations/baseline_random/86/bacc_obsInconclCases_baseline_random_development.png}
    \subcaption{}
    \label{fig:bacc_obsInconclCases_baseline_random_development}
  \end{subfigure}
  \hfill
  \begin{subfigure}{0.9\textwidth}
    \centering
    \includegraphics[width=0.9\textwidth]{content/figures/evaluations/baseline_random/86/bacc_obsInconclCases_concl_baseline_random_development.png}
    \subcaption{}
    \label{fig:bacc_obsInconclCases_concl_baseline_random_development}
  \end{subfigure}

  \caption{Evaluation of the CNN-RLT method on Test Set of Development Dataset.
  Balanced accuracy for a given mean percentage of observed inconclusive cases in the test set on 
  (a) both conclusive and inconclusive cases and (b) only conclusive cases. 
  Each of the mean percentages of observed inconclusive cases is associated with an inconclusive range (determined in the validation set). }
  \label{fig:bacc_obsInconclCases_baseline_random_development_full}
\end{figure}

%%%%%%%%%%%%%%%%%%%%%%%%%%%%%%%%%%%%%%%%%%%%%%%%%%%%%%%%%%%%%%%%%%%%%%%%%%%%%%%%%%%%%%%%%%%%%%%%%%%%%%%%%%%%%%%%%%%%%%%

% Evaluation on Independent datasets

% ---------------- PPMI ---------------------

% obsInconclCases_inconclCasesValid_baseline_random_ppmi
\begin{figure}[h]
\centering
\includegraphics[width=1.0\textwidth]{content/figures/evaluations/baseline_random/86/obsInconclCases_inconclCasesValid_baseline_random_ppmi.png}
\caption{Evaluation of the CNN-RLT method on PPMI Dataset.
Observed percentage of inconclusive cases in the PPMI dataset 
for a given set of percentages of inconclusive cases in the validation set (Development Dataset).
Each of the percentages of inconclusive cases in the validation set is associated 
with an inconclusive range (determined in the validation set).} 
\label{fig:obsInconclCases_inconclCasesValid_baseline_random_ppmi}
\end{figure} 


% bacc_obsInconclCases_baseline_random_ppmi_full
\begin{figure}[t]
\begin{subfigure}{0.9\textwidth}
  \centering
  \includegraphics[width=0.9\textwidth]{content/figures/evaluations/baseline_random/86/bacc_obsInconclCases_baseline_random_ppmi.png}
  \subcaption{}
  \label{fig:bacc_obsInconclCases_baseline_random_ppmi}
\end{subfigure}
\hfill
\begin{subfigure}{0.9\textwidth}
  \centering
  \includegraphics[width=0.9\textwidth]{content/figures/evaluations/baseline_random/86/bacc_obsInconclCases_concl_baseline_random_ppmi.png}
  \subcaption{}
  \label{fig:bacc_obsInconclCases_concl_baseline_random_ppmi}
\end{subfigure}

\caption{Evaluation of the CNN-RLT method on PPMI Dataset.
Balanced accuracy for a given mean percentage of inconclusive cases observed in the PPMI dataset on 
(a) both conclusive and inconclusive cases and (b) only conclusive cases. 
Each of the mean percentages of observed inconclusive cases is associated 
with an inconclusive range (determined in the validation set). }
\label{fig:bacc_obsInconclCases_baseline_random_ppmi_full}
\end{figure}



% -------------- MPH -----------------


% obsInconclCases_inconclCasesValid_baseline_random_mph
\begin{figure}[h]
\centering
\includegraphics[width=1.0\textwidth]{content/figures/evaluations/baseline_random/86/obsInconclCases_inconclCasesValid_baseline_random_mph.png}
\caption{Evaluation of the CNN-RLT method on MPH Dataset.
Observed percentage of inconclusive cases in the MPH dataset 
for a given set of percentages of inconclusive cases in the validation set (Development Dataset).
Each of the percentages of inconclusive cases in the validation set is associated 
with an inconclusive range (determined in the validation set).} 
\label{fig:obsInconclCases_inconclCasesValid_baseline_random_mph}
\end{figure} 


% bacc_obsInconclCases_baseline_random_mph_full
\begin{figure}[t]
\begin{subfigure}{0.9\textwidth}
  \centering
  \includegraphics[width=0.9\textwidth]{content/figures/evaluations/baseline_random/86/bacc_obsInconclCases_baseline_random_mph.png}
  \subcaption{}
  \label{fig:bacc_obsInconclCases_baseline_random_mph}
\end{subfigure}
\hfill
\begin{subfigure}{0.9\textwidth}
  \centering
  \includegraphics[width=0.9\textwidth]{content/figures/evaluations/baseline_random/86/bacc_obsInconclCases_concl_baseline_random_mph.png}
  \subcaption{}
  \label{fig:bacc_obsInconclCases_concl_baseline_random_mph}
\end{subfigure}

\caption{Evaluation of the CNN-RLT method on MPH Dataset.
Balanced accuracy for a given mean percentage of inconclusive cases observed in the MPH dataset on 
(a) both conclusive and inconclusive cases and (b) only conclusive cases. 
Each of the mean percentages of observed inconclusive cases is associated 
with an inconclusive range (determined in the validation set). }
\label{fig:bacc_obsInconclCases_baseline_random_mph_full}
\end{figure}


\subsection{Evaluation of Regression-based Method}
\label{subsec:eval_regression}


%%%%%%%%%%%%%%%%%%%%%%%%%%%%%%%%%%%%%%%%%%%%%%%%%%%%%%%%%%%%%%%%%%%%%%%%%%%%%%%%%%%%%%%%%%%%%%%%%%%%%%%%%%%%%%%%%%%%%%%

% Evaluation on Development dataset


\begin{table}[ht]
  \caption{Evaluation of the CNN-Regression method on Development dataset. 
  For evaluation, the natural sigmoid cutoff $0.5$ was used.}
  \centering
  \begin{tabular}{llll}
      \hline
                        & train set         & validation set      & test set             \\
      \hline
      Balanced Accuracy & 0.982+/-0.003   &  0.977+/-0.006    &  0.975+/-0.006 \\
      Accuracy          & 0.980+/-0.003     &   0.977+/-0.007   &  0.976+/-0.006  \\
      Sensitivity       &  1.000+/-0.000   &   0.983+/-0.009   &  0.961+/-0.011 \\
      Specificity       &   0.963+/-0.005  &   0.972+/-0.009 &   0.988+/-0.008 \\
      PPV               &  0.960+/-0.005    &   0.967+/-0.011  &  0.986+/-0.009  \\
      NPV               &  1.000+/-0.000  &   0.985+/-0.008   & 0.967+/-0.010 \\
      \hline
      AUC-ROC          &  \multicolumn{3}{c}{0.998+/-0.001}  \\
      \hline
  \end{tabular}
 \label{t1:cnn_regression_perf_eval_table}
\end{table}


% regression_percInconclCases_development
\begin{figure}[t]
  \centering
  \includegraphics[width=1.0\textwidth]{content/figures/evaluations/regression/86/sigmoid_percInconclCases_regression_development.png}
  \caption{Evaluation of the CNN-Regression method on Test Set of Development Dataset. 
  Determined upper and lower bounds of the inconclusive interval as a function of the percentage of inconclusive cases.} 
  \label{fig:regression_percInconclCases_development}
\end{figure}


% obsInconclCases_inconclCasesValid_regression_development
\begin{figure}[h]
  \centering
  \includegraphics[width=1.0\textwidth]{content/figures/evaluations/regression/86/obsInconclCases_inconclCasesValid_regression_development.png}
  \caption{Evaluation of the CNN-Regression method on Test Set of Development Dataset.
  Observed percentage of inconclusive cases in the test set 
  for a given set of percentages of inconclusive cases in the validation set.
  Each of the percentages of inconclusive cases in the validation set is associated 
  with an inconclusive range (determined in the validation set).} 
  \label{fig:obsInconclCases_inconclCasesValid_regression_development}
\end{figure} 


% bacc_obsInconclCases_regression_development_full
\begin{figure}[t]
  \begin{subfigure}{0.9\textwidth}
    \centering
    \includegraphics[width=0.9\textwidth]{content/figures/evaluations/regression/86/bacc_obsInconclCases_regression_development.png}
    \subcaption{}
    \label{fig:bacc_obsInconclCases_regression_development}
  \end{subfigure}
  \hfill
  \begin{subfigure}{0.9\textwidth}
    \centering
    \includegraphics[width=0.9\textwidth]{content/figures/evaluations/regression/86/bacc_obsInconclCases_concl_regression_development.png}
    \subcaption{}
    \label{fig:bacc_obsInconclCases_concl_regression_development}
  \end{subfigure}

  \caption{Evaluation of the CNN-Regression method on Test Set of Development Dataset.
  Balanced accuracy for a given mean percentage of observed inconclusive cases in the test set on 
  (a) both conclusive and inconclusive cases and (b) only conclusive cases. 
  Each of the mean percentages of observed inconclusive cases is associated with an inconclusive range (determined in the validation set). }
  \label{fig:bacc_obsInconclCases_regression_development_full}
\end{figure}

%%%%%%%%%%%%%%%%%%%%%%%%%%%%%%%%%%%%%%%%%%%%%%%%%%%%%%%%%%%%%%%%%%%%%%%%%%%%%%%%%%%%%%%%%%%%%%%%%%%%%%%%%%%%%%%%%%%%%%%

% Evaluation on Independent datasets

% ---------------- PPMI ---------------------

% obsInconclCases_inconclCasesValid_regression_ppmi
\begin{figure}[h]
\centering
\includegraphics[width=1.0\textwidth]{content/figures/evaluations/regression/86/obsInconclCases_inconclCasesValid_regression_ppmi.png}
\caption{Evaluation of the CNN-Regression method on PPMI Dataset.
Observed percentage of inconclusive cases in the PPMI dataset 
for a given set of percentages of inconclusive cases in the validation set (Development Dataset).
Each of the percentages of inconclusive cases in the validation set is associated 
with an inconclusive range (determined in the validation set).} 
\label{fig:obsInconclCases_inconclCasesValid_regression_ppmi}
\end{figure} 


% bacc_obsInconclCases_regression_ppmi_full
\begin{figure}[t]
\begin{subfigure}{0.9\textwidth}
  \centering
  \includegraphics[width=0.9\textwidth]{content/figures/evaluations/regression/86/bacc_obsInconclCases_regression_ppmi.png}
  \subcaption{}
  \label{fig:bacc_obsInconclCases_regression_ppmi}
\end{subfigure}
\hfill
\begin{subfigure}{0.9\textwidth}
  \centering
  \includegraphics[width=0.9\textwidth]{content/figures/evaluations/regression/86/bacc_obsInconclCases_concl_regression_ppmi.png}
  \subcaption{}
  \label{fig:bacc_obsInconclCases_concl_regression_ppmi}
\end{subfigure}

\caption{Evaluation of the CNN-Regression method on PPMI Dataset.
Balanced accuracy for a given mean percentage of inconclusive cases observed in the PPMI dataset on 
(a) both conclusive and inconclusive cases and (b) only conclusive cases. 
Each of the mean percentages of observed inconclusive cases is associated 
with an inconclusive range (determined in the validation set). }
\label{fig:bacc_obsInconclCases_regression_ppmi_full}
\end{figure}



% -------------- MPH -----------------


% obsInconclCases_inconclCasesValid_regression_mph
\begin{figure}[h]
\centering
\includegraphics[width=1.0\textwidth]{content/figures/evaluations/regression/86/obsInconclCases_inconclCasesValid_regression_mph.png}
\caption{Evaluation of the CNN-Regression method on MPH Dataset.
Observed percentage of inconclusive cases in the MPH dataset 
for a given set of percentages of inconclusive cases in the validation set (Development Dataset).
Each of the percentages of inconclusive cases in the validation set is associated 
with an inconclusive range (determined in the validation set).} 
\label{fig:obsInconclCases_inconclCasesValid_regression_mph}
\end{figure} 


% bacc_obsInconclCases_regression_mph_full
\begin{figure}[t]
\begin{subfigure}{0.9\textwidth}
  \centering
  \includegraphics[width=0.9\textwidth]{content/figures/evaluations/regression/86/bacc_obsInconclCases_regression_mph.png}
  \subcaption{}
  \label{fig:bacc_obsInconclCases_regression_mph}
\end{subfigure}
\hfill
\begin{subfigure}{0.9\textwidth}
  \centering
  \includegraphics[width=0.9\textwidth]{content/figures/evaluations/regression/86/bacc_obsInconclCases_concl_regression_mph.png}
  \subcaption{}
  \label{fig:bacc_obsInconclCases_concl_regression_mph}
\end{subfigure}

\caption{Evaluation of the CNN-Regression method on MPH Dataset.
Balanced accuracy for a given mean percentage of inconclusive cases observed in the MPH dataset on 
(a) both conclusive and inconclusive cases and (b) only conclusive cases. 
Each of the mean percentages of observed inconclusive cases is associated 
with an inconclusive range (determined in the validation set). }
\label{fig:bacc_obsInconclCases_regression_mph_full}
\end{figure}


\subsection{Comparative Analysis}
\label{subsec:compar_anal}


\subsubsection{Comparison of Performance on Test Split of Development Dataset}
\label{subsubsec:perf_comp_dev}


% Method Comparison of obsInconclCases_inconclCasesValid - Development test set
\begin{figure}[t]
  \begin{subfigure}{0.5\textwidth}
    \centering
    \includegraphics[width=1\textwidth]{content/figures/evaluations/sbr/43/obsInconclCases_inconclCasesValid_sbr_development.png}
    \subcaption{SBR method.}
  \end{subfigure}
  \hfill
  \begin{subfigure}{0.5\textwidth}
    \centering
    \includegraphics[width=1\textwidth]{content/figures/evaluations/pca_rfc/43/obsInconclCases_inconclCasesValid_pca_rfc_development.png}
    \subcaption{PCA-RFC method.}
  \end{subfigure}
  \hfill
  \begin{subfigure}{0.5\textwidth}
    \centering
    \includegraphics[width=1\textwidth]{content/figures/evaluations/baseline_majority/43/obsInconclCases_inconclCasesValid_baseline_majority_development.png}
    \subcaption{CNN method - MVT}
  \end{subfigure}
  \hfill
  \begin{subfigure}{0.5\textwidth}
    \centering
    \includegraphics[width=1\textwidth]{content/figures/evaluations/baseline_random/43/obsInconclCases_inconclCasesValid_baseline_random_development.png}
    \subcaption{CNN method - RLT}
  \end{subfigure}
  \hfill
  \begin{subfigure}{0.5\textwidth}
    \centering
    \includegraphics[width=1\textwidth]{content/figures/evaluations/regression/43/obsInconclCases_inconclCasesValid_regression_development.png}
    \subcaption{CNN method - Regression}
  \end{subfigure}

  \caption{Comparison of different methods on test set of development data. Interconclusive interval matching.}
  \label{fig:test_interval_match_dev}
\end{figure}

% Method Comparison of bacc-obsInconclCases-concl - Development test set
\begin{figure}[t]
  \begin{subfigure}{0.5\textwidth}
    \centering
    \includegraphics[width=1\textwidth]{content/figures/evaluations/sbr/43/bacc_obsInconclCases_concl_sbr_development.png}
    \subcaption{SBR method.}
  \end{subfigure}
  \hfill
  \begin{subfigure}{0.5\textwidth}
    \centering
    \includegraphics[width=1\textwidth]{content/figures/evaluations/pca_rfc/43/bacc_obsInconclCases_concl_pca_rfc_development.png}
    \subcaption{PCA-RFC method.}
  \end{subfigure}
  \hfill
  \begin{subfigure}{0.5\textwidth}
    \centering
    \includegraphics[width=1\textwidth]{content/figures/evaluations/baseline_majority/43/bacc_obsInconclCases_concl_baseline_majority_development.png}
    \subcaption{CNN method - MVT}
  \end{subfigure}
  \hfill
  \begin{subfigure}{0.5\textwidth}
    \centering
    \includegraphics[width=1\textwidth]{content/figures/evaluations/baseline_random/43/bacc_obsInconclCases_concl_baseline_random_development.png}
    \subcaption{CNN method - RLT}
  \end{subfigure}
  \hfill
  \begin{subfigure}{0.5\textwidth}
    \centering
    \includegraphics[width=1\textwidth]{content/figures/evaluations/regression/43/bacc_obsInconclCases_concl_regression_development.png}
    \subcaption{CNN method - Regression}
  \end{subfigure}

  \caption{Comparison of different methods on test set of development data.}
  \label{fig:test_dev}
\end{figure}



\subsubsection{Comparison of Performance on Independent datasets}
\label{subsubsec:perf_comp_indep}


% ------ PPMI --------


% Method Comparison of obsInconclCases_inconclCasesValid - PPMI
\begin{figure}[t]
  \begin{subfigure}{0.5\textwidth}
    \centering
    \includegraphics[width=1\textwidth]{content/figures/evaluations/sbr/43/obsInconclCases_inconclCasesValid_sbr_ppmi.png}
    \subcaption{SBR method.}
  \end{subfigure}
  \hfill
  \begin{subfigure}{0.5\textwidth}
    \centering
    \includegraphics[width=1\textwidth]{content/figures/evaluations/pca_rfc/43/obsInconclCases_inconclCasesValid_pca_rfc_ppmi.png}
    \subcaption{PCA-RFC method.}
  \end{subfigure}
  \hfill
  \begin{subfigure}{0.5\textwidth}
    \centering
    \includegraphics[width=1\textwidth]{content/figures/evaluations/baseline_majority/43/obsInconclCases_inconclCasesValid_baseline_majority_ppmi.png}
    \subcaption{CNN method - MVT}
  \end{subfigure}
  \hfill
  \begin{subfigure}{0.5\textwidth}
    \centering
    \includegraphics[width=1\textwidth]{content/figures/evaluations/baseline_random/43/obsInconclCases_inconclCasesValid_baseline_random_ppmi.png}
    \subcaption{CNN method - RLT}
  \end{subfigure}
  \hfill
  \begin{subfigure}{0.5\textwidth}
    \centering
    \includegraphics[width=1\textwidth]{content/figures/evaluations/regression/43/obsInconclCases_inconclCasesValid_regression_ppmi.png}
    \subcaption{CNN method - Regression}
  \end{subfigure}

  \caption{Comparison of different methods on PPMI dataset. Interconclusive interval matching.}
  \label{fig:test_interval_match_ppmi}
\end{figure}


% Method Comparison of bacc_obsInconclCases_concl - PPMI test set
\begin{figure}[t]
  \begin{subfigure}{0.5\textwidth}
    \centering
    \includegraphics[width=1\textwidth]{content/figures/evaluations/sbr/43/bacc_obsInconclCases_concl_sbr_ppmi.png}
    \subcaption{SBR method.}
  \end{subfigure}
  \hfill
  \begin{subfigure}{0.5\textwidth}
    \centering
    \includegraphics[width=1\textwidth]{content/figures/evaluations/pca_rfc/43/bacc_obsInconclCases_concl_pca_rfc_ppmi.png}
    \subcaption{PCA-RFC method.}
  \end{subfigure}
  \hfill
  \begin{subfigure}{0.5\textwidth}
    \centering
    \includegraphics[width=1\textwidth]{content/figures/evaluations/baseline_majority/43/bacc_obsInconclCases_concl_baseline_majority_ppmi.png}
    \subcaption{CNN method - MVT}
  \end{subfigure}
  \hfill
  \begin{subfigure}{0.5\textwidth}
    \centering
    \includegraphics[width=1\textwidth]{content/figures/evaluations/baseline_random/43/bacc_obsInconclCases_concl_baseline_random_ppmi.png}
    \subcaption{CNN method - RLT}
  \end{subfigure}
  \hfill
  \begin{subfigure}{0.5\textwidth}
    \centering
    \includegraphics[width=1\textwidth]{content/figures/evaluations/regression/43/bacc_obsInconclCases_concl_regression_ppmi.png}
    \subcaption{CNN method - Regression}
  \end{subfigure}

  \caption{Comparison of different methods on PPMI dataset.}
  \label{fig:test_ppmi}
\end{figure}


% ------ MPH --------

% Method Comparison of obsInconclCases_inconclCasesValid - MPH
\begin{figure}[t]
  \begin{subfigure}{0.5\textwidth}
    \centering
    \includegraphics[width=1\textwidth]{content/figures/evaluations/sbr/43/obsInconclCases_inconclCasesValid_sbr_mph.png}
    \subcaption{SBR method.}
  \end{subfigure}
  \hfill
  \begin{subfigure}{0.5\textwidth}
    \centering
    \includegraphics[width=1\textwidth]{content/figures/evaluations/pca_rfc/43/obsInconclCases_inconclCasesValid_pca_rfc_mph.png}
    \subcaption{PCA-RFC method.}
  \end{subfigure}
  \hfill
  \begin{subfigure}{0.5\textwidth}
    \centering
    \includegraphics[width=1\textwidth]{content/figures/evaluations/baseline_majority/43/obsInconclCases_inconclCasesValid_baseline_majority_mph.png}
    \subcaption{CNN method - MVT}
  \end{subfigure}
  \hfill
  \begin{subfigure}{0.5\textwidth}
    \centering
    \includegraphics[width=1\textwidth]{content/figures/evaluations/baseline_random/43/obsInconclCases_inconclCasesValid_baseline_random_mph.png}
    \subcaption{CNN method - RLT}
  \end{subfigure}
  \hfill
  \begin{subfigure}{0.5\textwidth}
    \centering
    \includegraphics[width=1\textwidth]{content/figures/evaluations/regression/43/obsInconclCases_inconclCasesValid_regression_mph.png}
    \subcaption{CNN method - Regression}
  \end{subfigure}

  \caption{Comparison of different methods on MPH dataset. Interconclusive interval matching.}
  \label{fig:test_interval_match_mph}
\end{figure}


% Method Comparison of bacc_obsInconclCases_concl - MPH test set
\begin{figure}[t]
  \begin{subfigure}{0.5\textwidth}
    \centering
    \includegraphics[width=1\textwidth]{content/figures/evaluations/sbr/43/bacc_obsInconclCases_concl_sbr_mph.png}
    \subcaption{SBR method.}
  \end{subfigure}
  \hfill
  \begin{subfigure}{0.5\textwidth}
    \centering
    \includegraphics[width=1\textwidth]{content/figures/evaluations/pca_rfc/43/bacc_obsInconclCases_concl_pca_rfc_mph.png}
    \subcaption{PCA-RFC method.}
  \end{subfigure}
  \hfill
  \begin{subfigure}{0.5\textwidth}
    \centering
    \includegraphics[width=1\textwidth]{content/figures/evaluations/baseline_majority/43/bacc_obsInconclCases_concl_baseline_majority_mph.png}
    \subcaption{CNN method - MVT}
  \end{subfigure}
  \hfill
  \begin{subfigure}{0.5\textwidth}
    \centering
    \includegraphics[width=1\textwidth]{content/figures/evaluations/baseline_random/43/bacc_obsInconclCases_concl_baseline_random_mph.png}
    \subcaption{CNN method - RLT}
  \end{subfigure}
  \hfill
  \begin{subfigure}{0.5\textwidth}
    \centering
    \includegraphics[width=1\textwidth]{content/figures/evaluations/regression/43/bacc_obsInconclCases_concl_regression_mph.png}
    \subcaption{CNN method - Regression}
  \end{subfigure}

  \caption{Comparison of different methods on MPH dataset.}
  \label{fig:test_mph}
\end{figure}



% AUC bAcc comparison of different methods and dataset
\begin{figure}[h]
  \centering
  \includegraphics[width=1.0\textwidth]{content/figures/evaluations/auc_main_comparison.png}
  \caption{Comparison of AUC achieved by each method on different test data. 
  The AUC was calculated for the mean balanced accuracy over the percentage of inconclusive cases 
  in the considered test set.} 
  \label{fig:auc_comparison_methods_data}
\end{figure} 