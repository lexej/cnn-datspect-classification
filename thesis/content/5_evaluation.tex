\section{Evaluation}
\label{sec:evaluation}

The preceding chapters have detailed the research methodology, data collection and sources, and the application of classification techniques 
to address the research questions posed in this study. 

This chapter embarks on the evaluation of the research results, focusing on the performance and effectiveness of the methods employed, 
and the attainment of the research objectives.

The structure of this chapter has been designed to systematically lead readers through the assessment process. 
It commences with a discussion of the research objectives that serve as the focal points for subsequent evaluations. 
Following this, a comprehensive examination of performance metrics is conducted, with an emphasis on their 
significance within the context of this research, providing detailed insights into the criteria employed 
to evaluate research outcomes.
The core of this chapter subsequently unveils the experimental results encompassing various test datasets 
and classification methods. 
These findings are presented using graphical representations and supported by a range of statistical measures.
The chapter culminates with a comparative analysis, which seeks to assess and contrast the effectiveness and 
limitations of the research methods employed.

\subsection{Research Objectives}
\label{subsec:research_objectives}

% TODO

% Objective 1: Evaluation of Methods on Test set of dev data

%   Objective 1.1: To compare the "performance" of the methods on Test set of dev data.
%   Objective 1.2: To assess the suitability of these methods in unseen real-world scenarios


% Objective 2: Evaluation of Methods on PPMI and MPH dataset
%				-> understand the impact of varying dataset characteristics by evaluating on


\subsection{Evaluation Metrics}
\label{subsec:determinationInconcl}

In the following the performance metrics used for the evaluation of the different classification methods 
are explained in more detail.

First the $\text{mean} \pm \text{SD}$ (standard deviation) of the following measures were calculated across
the different random splits for each classification approach: 
Balanced accuracy, accuracy, sensitivity, specificity, PPV and NPV.
The natural cutoff of 0.5 was used for each classification approach except the SBR (for SBR: determined optimal cutoff).

% AUC of bal_acc vs. %inconclusive_cases; determination of inconclusive ranges for each %inconclusive_cases
The main performance metric used in this work to evaluate and compare the classification approaches was 
the area under the curve (AUC) of balanced accuracy in conclusive cases (in test set) as a function of 
the proportion of inconclusive cases (in test set).
In this context, inconclusive cases are defined as cases predicted within an inconclusive interval 
(defined by lower and upper bound), while conclusive cases are those predicted outside this interval.
The set of percentages of inconclusive cases considered ranged from 0.2\% to 20.0\%, increasing in increments of 0.2\%.

% Determination of inconclusive intervals given set of percentages of inconclusive cases
To compute the balanced accuracy in conclusive cases for a certain percentage of inconclusive cases, 
the corresponding inconclusive interval was first determined 
for each element in the considered set of percentages of inconclusive cases.
The determination of the inconclusive interval was exclusively performed using the validation set 
for each random split and classification approach independently.
The lower and upper bounds of each inconclusive interval were independently determined
to ensure a similar number of inconclusive cases both below and above the pre-defined cutoff (decision threshold).
For the CNN-based classification approaches (described in Section~\ref{subsec:cnn_based_classification}) and the 
multivariate benchmark (described in Section~\ref{subsec:pca_rfc}) the natural cutoff of 0.5 was used.
For the SBR-based univariate benchmark (described in Section~\ref{subsec:sbr}), 
the optimal cutoff on the SBR obtained by applying the Youden criterion~\citep{Youden1950} using ROC analysis was used.


% Stability of inconclusive interval -> Ralph fragen: Stability of inconclusive interval on validation or test set?
Furthermore, the stability of the inconclusive interval as a function of the proportion of inconclusive cases is evaluated.
Therefore the $\text{mean} \pm \text{SD}$ of the determined upper and lower bounds of the inconclusive interval, 
calculated from the proportion of inconclusive cases in the validation set across different random splits, 
were plotted against the corresponding proportion of inconclusive cases.
The functions were created separately for each classification approach.

% Proportion of inconclusive cases in test set vs. 
Also the observed $\text{mean} \pm \text{SD}$ proportion of inconclusive cases in the test set is plotted 
against the proportion of inconclusive cases in the validation set.


% TODO
Erklärung Algorithmus zum Inconclusive interval bestimmung hier / zu Methods? 




\subsection{Evaluation on Test Split of Development Dataset}
\label{subsec:eval_dev}


\subsubsection{Evaluation of SBR Method}
\label{subsubsec:eval_sbr}


% SBR - sbr_percInconclCases_development
\begin{figure}[t]
    \centering
    \includegraphics[width=1.0\textwidth]{content/figures/evaluations/sbr/sbr_percInconclCases_development.png}
    \caption{Evaluation of the SBR method. Determined upper and lower bounds of the inconclusive interval 
    as a function of the percentage of inconclusive cases.} 
    \label{fig:sbr_percInconclCases_development}
\end{figure} 


% SBR - obsInconclCases_inconclCasesValid_sbr_development
\begin{figure}[h]
    \centering
    \includegraphics[width=1.0\textwidth]{content/figures/evaluations/sbr/obsInconclCases_inconclCasesValid_sbr_development.png}
    \caption{Evaluation of the SBR method. Observed percentage of inconclusive cases in the test set 
    for a given set of percentages of inconclusive cases in the validation set.
    Each of the percentages of inconclusive cases in the validation set is associated 
    with an inconclusive range (determined in the validation set).} 
    \label{fig:obsInconclCases_inconclCasesValid_sbr_development}
\end{figure} 


% SBR - bacc_obsInconclCases_sbr_development_full
\begin{figure}[t]
    \begin{subfigure}{0.9\textwidth}
      \centering
      \includegraphics[width=0.9\textwidth]{content/figures/evaluations/sbr/bacc_obsInconclCases_sbr_development.png}
      \subcaption{}
      \label{fig:bacc_obsInconclCases_sbr_development}
    \end{subfigure}
    \hfill
    \begin{subfigure}{0.9\textwidth}
      \centering
      \includegraphics[width=0.9\textwidth]{content/figures/evaluations/sbr/bacc_obsInconclCases_concl_sbr_development.png}
      \subcaption{}
      \label{fig:bacc_obsInconclCases_concl_sbr_development}
    \end{subfigure}

    \caption{Evaluation of the SBR method. Balanced accuracy for a given mean percentage of observed inconclusive cases in the test set on 
    (a) both conclusive and inconclusive cases and (b) only conclusive cases. 
    Each of the mean percentages of observed inconclusive cases is associated with an inconclusive range (determined in the validation set). }
    \label{fig:bacc_obsInconclCases_sbr_development_full}
\end{figure}





\subsubsection{Evaluation of Random Forest Method}
\label{subsubsec:eval_rfc}




\subsubsection{Evaluation of MVT-based Method}
\label{subsubsec:eval_mvt}


\subsubsection{Evaluation of RLT-based Method}
\label{subsubsec:eval_rlt}


\subsubsection{Evaluation of Regression-based Method}
\label{subsubsec:eval_regression}



\subsection{Evaluation on Independent Datasets}
\label{subsec:eval_independent}

% Per method: PPMI and MPH 

\subsubsection{Evaluation of Univariate and Multivariate Benchmark Methods}
\label{subsubsec:eval_dev_indep}


\subsubsection{Evaluation of MVT-based Method}
\label{subsubsec:eval_mvt_indep}


\subsubsection{Evaluation of RLT-based Method}
\label{subsubsec:eval_rlt_indep}


\subsubsection{Evaluation of Regression-based Method}
\label{subsubsec:eval_regression_indep}



\subsection{Comparative Analysis}
\label{subsec:compar_anal}

\subsubsection{Comparison of Performance on Test Split of Development Dataset}
\label{subsubsec:perf_comp_dev}


\subsubsection{Comparison of Performance on Independent datasets}
\label{subsubsec:perf_comp_indep}


