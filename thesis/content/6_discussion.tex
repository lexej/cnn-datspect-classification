\section{Discussion}
\label{sec:discussion}

\subsection{Interpretation of Results}

The primary research hypothesis proposed that a CNN trained using the RLT strategy
is more effective in identifying inconclusive cases compared to the MVT strategy.
The main metric proposed to evaluate and compare the performance was the
AUC of balanced accuracy on conclusive cases over the percentage of observed inconclusive cases (PIncObs) in the test set.
The performance results summarized in Figure~\ref{fig:auc_comparison_methods_data} demonstrate 
that the RLT strategy leads to slightly better performance (higher by $0.05-0.1\%$) on the development data test set 
and the external PPMI test dataset.
On the internal MPH test dataset the performance of the CNN using RLT strategy is higher by $0.4\%$ compared to the MVT strategy.
Since the MPH dataset cases exhibit better spatial resolution than the development dataset and PPMI dataset cases
and thus are more complex and difficult to classify 
the clear superiority of the RLT strategy on this test dataset is particularly remarkable.
Hence the findings support the primary hypothesis of this work.

Two secondary hypotheses were also put to test in this work.
The first hypothesis was the superiority in performance of CNN-based classification methods compared to 
conventional baseline methods in terms of the main metric (AUC of balanced accuracy over PIncObs).
The results shown in Figure~\ref{fig:auc_comparison_methods_data} show that CNN-based classification methods
have a consistent AUC performance advantage over the SBR univariate baseline method across all test datasets, 
achieving approximately a 2\% higher AUC result.
The AUC performance of the multivariate baseline method PCA-RFC closely approaches that of the CNN-based methods 
on the development data test set and PPMI test dataset.
However, on the more complex MPH test dataset, the AUC performance of the PCA-RFC method is over 3\% lower
compared to that of the CNN-based methods.
The second hypothesis stated that the CNN-based classification methods are 
more robust regarding varying image characteristics and thus exhibit better generalizability compared to baseline methods.
The CNN-based methods outperform both baseline methods on the PPMI and MPH test datasets which differ in image characteristics.
The performance advantage of the CNN-based methods is more prominent on the more complex MPH dataset.
Consequently, both secondary hypotheses are confirmed by the findings.


\subsection{Practical Implications}

The findings of the study have practical implications for the classification of DAT-SPECT images.
First, the study shows that random label selection as a ground-truth label selection strategy 
can lead to better performance results compared to the majority vote strategy 
when training a CNN classifier for Parkinson's disease diagnosis.
Second, the mean AUC of balanced accuracy on conclusive cases over the mean percentage of observed inconclusive cases (PIncObs)
can be used as a metric for comparison of the actual underlying decision confidence of 
different binary classification models.
The metric decouples the classification model performance from the arbitrarily chosen inconclusive interval bounds.
Also, the balanced accuracy on conclusive cases over PIncObs can be used to decide for the model that 
produces the least relative amount of inconclusive cases for a specified target balanced accuracy.
The applicability of the metric extends beyond DAT-SPECT classification to general binary classification problems.
Third, the results once again confirm the superiority of CNN-based methods for DAT-SPECT classification compared to the
widely adopted SBR method in clinical practice.


\subsection{Limitations of the Study}

\subsubsection{Metric}

There are several limitations to be considered that may impact the validity of the applied methods and results.
The main metric used to compare the model performance, the AUC of mean balanced accuracy over mean PIncObs, 
depends on a set of inconclusive intervals determined within the validation set of the development dataset 
for each classification method and randomization individually.
Since the balanced accuracy and PIncObs are averaged across the results for each random split 
the reliability of the metric is affected by the standard deviation of both variables across the random splits.
To enhance the reliability of the metric a higher number of random splits can be used.
Also, the resolution of the balanced accuracy over PIncObs decreases as the density of test set 
predictions around the cutoff increases in comparison to the validation set predictions.
Finally the metric may be less intuitively understandable and requires more expertise when interpreting the results
when compared to standard metrics such as balanced accuracy and ROC-AUC.


\subsubsection{Site-Specific Development Data}

To increase the heterogeneity of the training data and enhance the generalizability 
of the classification models the development dataset was augmented as described in Section~\ref{subsec:augment}.
However the results, presented in Figure~\ref{fig:auc_comparison_methods_data}, 
show that AUC performance on the MPH test across all considered classification methods, 
is significantly lower than on the test set of the development dataset.
The reason for that can be a site-specific bias introduced by training on data from one site (development dataset).
Future research efforts should aim to include development data from multiple sites 
to enhance the overall robustness of the classification methods and results.


%Address any limitations in the methodology or data that might have affected the results.
%Discuss challenges faced during the experiments.


%\subsection{Future Research}
%Future research ->  directions for future research based on the outcomes and limitations of the current study





% Conclusion -> key points discussed in the evaluation and discussion chapters.