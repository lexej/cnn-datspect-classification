\section{Discussion}
\label{sec:discussion}


The primary research hypothesis proposed that a CNN trained using the RLT strategy
is more effective in identifying inconclusive cases compared to the MVT strategy.
The main metric proposed to evaluate and compare the performance was the
AUC of balanced accuracy on conclusive cases over the percentage of observed inconclusive cases (PIncObs) in the test set.
The performance results summarized in Figure~\ref{fig:auc_comparison_methods_data} demonstrate 
that the RLT strategy leads to slightly better performance (higher by $0.05-0.1\%$) on the development data test set 
and the external PPMI test dataset.
On the internal MPH test dataset the performance of the CNN using RLT strategy is higher by $0.4\%$ compared to the MVT strategy.
Since the MPH dataset cases exhibit better spatial resolution than the development dataset and PPMI dataset cases
and thus are more complex and difficult to classify 
the clear superiority of the RLT strategy on this test dataset is particularly remarkable.
Hence the findings support the primary hypothesis of this work.

Two secondary hypotheses were also put to test in this work.
The first hypothesis was the superiority in performance of CNN-based classification methods compared to 
conventional baseline methods in terms of the main metric (AUC of balanced accuracy over PIncObs).
The results shown in Figure~\ref{fig:auc_comparison_methods_data} show that CNN-based classification methods
have a consistent AUC performance advantage over the SBR univariate baseline method across all test datasets, 
achieving approximately a 2\% higher AUC result.
The AUC performance of the multivariate baseline method PCA-RFC closely approaches that of the CNN-based methods 
on the development data test set and PPMI test dataset.
However on the more complex MPH test dataset, the AUC performance of the PCA-RFC method is over 3\% lower
compared to that of the CNN-based methods.
The second hypothesis stated that the CNN-based classification methods are 
more robust regarding varying image characteristics and thus exhibit better generalizability compared to baseline methods.
The CNN-based methods outperform both baseline methods on the PPMI and MPH test datasets which differ in image characteristics.
The performance advantage of the CNN-based methods is more prominent on the more complex MPH dataset.
Consequently both secondary hypotheses are confirmed by the findings.





Some Limitations: 
- bACC-AUC metric may be less human interpretable compared to the standard AUC, 
however when usefull to compare performance of classification models
- Higher SD across random splits is a weak point of the bAcc-AUC metric..


%Interpretation of Results:
%Analyze the findings in the context of the research questions and objectives.

%Comparison with Baseline Methods:
%%Discuss how Method 1, Method 2, and Method 3 compare with Baseline A and Baseline B.

%Identification of Patterns/Trends:
%Identify any patterns or trends in the results and discuss their significance.

%Addressing Research Questions:
%Discuss how well the research questions were answered based on the results.

%Limitations and Challenges:
%Address any limitations in the methodology or data that might have affected the results.
%Discuss challenges faced during the experiments.

%Implications and Applications:
%Discuss the practical implications of the results in real-world ML applications.

%Future Work:
%Propose directions for future research based on the outcomes and limitations of the current study.

%Conclusion:
%Summarize the key points discussed in the evaluation and discussion chapters.