\section{Methods}
\label{sec:methods}

\subsection{Development Data Preparation}

In the following, the data preparation techniques applied to the development dataset are explained in detail.

\subsubsection{Data Preprocessing}
\label{subsubsec:img_preprocess_dev}

% Preprocessing of development dataset

% TODO -> wo "as described previously" -> näher beschreiben

Individual DAT-SPECT images were stereotactically normalized to the anatomical space of the Montreal Neurological Institute (MNI) 
using the Normalize tool of the Statistical Parametric Mapping software package (version SPM12) and a set of custom DAT-SPECT templates 
representative of normal and different levels of Parkinson-typical reduction of striatal uptake as target [73]. 
The voxel size of the stereotactically normalized images was 2x2x2 mm$^{3}$. 
Intensity normalization was achieved by voxelwise scaling to the individual 75th percentile of the voxel intensity in a reference region 
comprising the whole brain without striata, thalamus, brainstem, cerebellum, and ventricles [74]. 
The resulting images are distribution volume (DVR) images. 
A 2-dimensional transversal DVR slab of 12mm thickness and 91x109 pixels with 2 mm edge length was obtained by averaging 6 transversal slices through the striatum [75]. 

\subsubsection{Data Augmentation}
\label{subsec:augment}

Data augmentation was applied to the development dataset to increase the heterogeneity of the data.
To enhance robustness across various attenuation correction and scatter correction methods, 
each image was generated both with and without applying attenuation and scatter corrections.
Also 3D-smoothing was employed for augmentation using an isotropic Gaussian kernel with various 
Full Width at Half Maximum (FWHM) values (FWHM = 10, 12, 14, 16, 18mm).
Thereby an augmented dataset of 20,880 images in total was constructed based on 1,740 cases.
An example of two cases augmented using the described techniques is depicted in Figure~\ref{fig:dev_dataset}.

\begin{figure}[t]
    \centering
    \colorbox{black}{%
     \includegraphics[width=0.9\textwidth]{content/figures/dev_dataset.png}%
     }
    \caption{Images obtained through augmentation of two sample cases from the development dataset. 
    Normal case (above) and case with reduced availability of dopamine transporters (DAT) in the striatum (below).} 
    \label{fig:dev_dataset}
  \end{figure} 

\subsubsection{Dataset Splitting}
\label{subsec:split}

% How was development dataset split
The augmented development dataset was split into three subsets: train set (60\%), validation set (20\%) 
and test set (20\%).
While splitting the data it was ensured that the augmented images belonging to a concrete patient were put 
only into one subset.
Thereby inter-subset data leakage was prohibited.
Ten different random splits were created to train and test each of the methods.

\subsection{Univariate benchmark: Specific Binding Ratio}
\label{subsec:sbr}

% TODO -> wo "as described previously" -> näher beschreiben

The unilateral [$^{123}$I]FP-CIT specific binding ratio (SBR) in left and right putamen was obtained by hottest voxels analysis 
of the stereotactically normalized DVR image using large unilateral putamen masks predefined in MNI space as described previously [46]. 
The minimum of both hemispheres was used for the analyses.


\subsection{Multivariate benchmark: PCA-enhanced Random Forest}
\label{subsec:pca_rfc}


\subsection{CNN-based classification - MVT and RLT}
\label{subsec:cnn_based_classification}



