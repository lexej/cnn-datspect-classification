\section{Methods}
\label{sec:methods}

\subsection{Development Data Preparation}

In the following, the data preparation techniques applied to the development dataset are explained in detail.

\subsubsection{Data Preprocessing}
\label{subsubsec:img_preprocess_dev}

% Preprocessing of development dataset

% TODO -> wo "as described previously" -> näher beschreiben

Individual DAT-SPECT images were stereotactically normalized to the anatomical space of the Montreal Neurological Institute 
using the Normalize tool of the Statistical Parametric Mapping software package (version SPM12) and a set of custom DAT-SPECT templates 
representative of normal and different levels of Parkinson-typical reduction of striatal uptake as target [73]. 
The voxel size of the stereotactically normalized images was 2x2x2 mm$^{3}$. 
Intensity normalization was achieved by voxelwise scaling to the individual 75th percentile of the voxel intensity in a reference region 
comprising the whole brain without striata, thalamus, brainstem, cerebellum, and ventricles [74]. 
The resulting images are distribution volume (DVR) images. 
A 2-dimensional transversal DVR slab of 12mm thickness and 91x109 pixels with 2 mm edge length was obtained by averaging 6 transversal slices through the striatum [75]. 

\subsubsection{Data Augmentation}
\label{subsec:augment}

%   Data augmentation - 
Data augmentation was applied to the development dataset to increase the heterogeneity of the data.


\subsubsection{Dataset Splitting}
\label{subsec:split}

% How was development dataset split


\subsection{Univariate benchmark: Specific Binding Ratio}
\label{subsec:sbr}

% TODO -> wo "as described previously" -> näher beschreiben

The unilateral [$^{123}$I]FP-CIT specific binding ratio (SBR) in left and right putamen was obtained by hottest voxels analysis 
of the stereotactically normalized DVR image using large unilateral putamen masks predefined in MNI space as described previously [46]. 
The minimum of both hemispheres was used for the analyses.


\subsection{Multivariate benchmark: PCA-enhanced Random Forest}
\label{subsec:pca_rfc}


\subsection{CNN-based classification - MVT and RLT}
\label{subsec:cnn_based_classification}



