\section{Conclusion}

This work contributed to a better understanding of the performance differences 
resulting from the usage of either random label selection or majority vote selection
as label selection strategies for training convolutional neural networks.
The results showed a slight performance advantage on test data for the random label selection strategy 
concerning the proposed AUC-bACC metric.
To justify additional costs for obtaining multiple ground truth assessments for DAT-SPECT images
the significance of the performance difference on more diverse real-world data 
has to be further investigated in future work.
The proposed AUC-bACC performance metric allows to decide for a concrete classification method among different methods
considering both the cost of manual inspection of inconclusive DAT-SPECT cases 
by physicians and the classification performance on conclusive cases.
Both aspects are crucial for an automatic DAT-SPECT image classification method to be useful in clinical practice.
The metric also allows for the derivation of an operating point for a binary DAT-SPECT classifier 
given a target balanced accuracy.
Further research has to be conducted to assess the robustness of the AUC-bACC metric 
and the statistical significance of the produced results.
The study further confirmed the performance advantage of CNN based DAT-SPECT classification compared to benchmark methods.
The higher robustness of CNN methods is particularly prominent when evaluating on unseen DAT-SPECT images 
with higher spatial resolution.

% Recap the main findings and their implications.

% Emphasize the contributions of the study to the field of Machine Learning.

% Discuss the broader significance of the research in the context of real-world applications.

% Provide suggestions for practitioners and policymakers based on the findings.

% Revisit the limitations and propose potential ways to overcome them in future research.