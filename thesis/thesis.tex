% Version History
% 2020-10-27  	    Holger Graf
%                   Licence CC BY 4.0
%					Link: https://www.overleaf.com/latex/templates/thesis-template-microeconomics-at-fsu-jena/shqhkgcqtvsn
%					Accessed 18.01.2023
% 2023-02-01		Adapted by Ines Rieger
%					Licence CC BY 4.0
%					Lehrstuhl für Erklärbares Maschinelles Lernen
%					Universität Bamberg
% Please adapt this tex file for your thesis

\documentclass{xai-thesis}

\usepackage{graphicx}
\usepackage{setspace}
\usepackage{hyperref} % use \usepackage[hidelinks]{hyperref} to hide boxes
\usepackage[utf8]{inputenc} % depends on the font encoding that you are using
\usepackage[round]{natbib}

\input{math_commands.tex}

\begin{document}
% ----------------------------------------------------------------------------
% Details for the titlepage
% ----------------------------------------------------------------------------
\thesisTitle{Title of your Thesis}
\thesisType{Master Thesis} % 'Master Thesis' or 'Bachelor Thesis'
\thesisAuthor{Aleksej Kucerenko}
\thesisGrade{Master of Science in Applied Computer Science}
\thesisFirstSupervisor{The name of your first supervisor} % your supervising professor
\thesisSecondSupervisor{The name of your second supervisor, if applicable} % if you are supervised by additional advisors, e.g phd students at the chair or a  supervisor in a company (write company after the name of the supervisor)
\thesisDate{\today}

% Print titlepage
\thesisMakeTitle

% ----------------------------------------------------------------------------
% Abstract
% ----------------------------------------------------------------------------
\clearpage
\pagenumbering{roman}
\pagestyle{plain}

\subsection*{Abstract}
Short summary of your thesis (max. 1 page) \ldots

\clearpage
\subsection*{Abstract}
Kurze Zusammenfassung Ihrer Abschlussarbeit (max. 1 Seite) \ldots

% ----------------------------------------------------------------------------
% Acknowledgements
% ----------------------------------------------------------------------------
\clearpage
\subsection*{Acknowledgements}
If you want to thank anyone (optional) \ldots

% ----------------------------------------------------------------------------
% Table of contents
% ----------------------------------------------------------------------------
\clearpage
%\thispagestyle{empty}
\tableofcontents

% ----------------------------------------------------------------------------
% List of figures/tables/acronyms
% ----------------------------------------------------------------------------
\clearpage
\phantomsection
\addcontentsline{toc}{section}{List of Figures}
\listoffigures
% --------------------------
\clearpage
\phantomsection
\addcontentsline{toc}{section}{List of Tables}
\listoftables
% --------------------------
\clearpage
\phantomsection
\addcontentsline{toc}{section}{List of Acronyms}
\section*{List of Acronyms}
\begin{tabular}{@{}ll}
AI & Artificial Intelligence\\
\end{tabular}

% ----------------------------------------------------------------------------
% Notation
% ----------------------------------------------------------------------------
\clearpage
\section*{Notation}
\input{math_notation.tex}

% --------------------------

% ----------------------------------------------------------------------------
% Contents
% ----------------------------------------------------------------------------
\cleardoublepage
\pagestyle{headings}
\pagenumbering{arabic}
\setcounter{page}{1}

% Hint: It is advisable to split the thesis into several tex files organized by sections. You can use thesis.tex as the main file and include any other tex file with \input(introduction.tex)

\section{Introduction}
\label{sec:intro}

Some of your text.

\section{First Section}
\label{sec:first}


\subsection{First Subsection}
\label{subsec:first}


\subsection{Second Subsection}
\label{subsec:second}

Some more of your text. For citations, use the command \verb+\citep{lecun2015deep}+ which produces~\citep{lecun2015deep} or \verb+\cite{lecun2015deep}+ which produces~\cite{lecun2015deep}.


\section{Second Section}
\label{sec:second}

\subsection{Another Subsection}
\label{subsec:another}
Here is a reference to Table~\ref{t1:sample}. Figure~\ref{fig:xai_logo} shows \ldots


\begin{table}[ht]
	\caption{Sample table title}
	\centering
	\begin{tabular}{lll}
		\hline
		Name     & Description     & Size ($\mu$m) \\
		\hline
		Dendrite & Input terminal  & $\sim$100     \\
		Axon     & Output terminal & $\sim$10      \\
		Soma     & Cell body       & up to $10^6$  \\
		\hline
	\end{tabular}
   \label{t1:sample}
\end{table}


\begin{figure}[ht]
  \centering
   \includegraphics[width=.5\textwidth]{xaiLogo.png}
  \caption{Logo of the chair of Explainable Machine Learning}   
  \label{fig:xai_logo}
\end{figure}  

\subsection{Yet another Subsection}
\label{subsec:yetanother}

If you want to typeset formulas, there is the inline version $ y = x_1^{0.5} x_2^{0.5}$, centered like this
\[
y = x_1^{0.5} x_2^{0.5}
\]
or numbered:
\begin{equation}\label{eq:prod}
y = x_1^{0.5} x_2^{0.5}	
\end{equation}
so that you can refer to equation~\ref{eq:prod} in the text.

\subsection{Last Subsection}
\label{subsec:last}

\section{Conclusion}



% --------------------------
\clearpage
\begin{appendix}
	\section{Appendix}
	If needed for supplementary material, such as detailed description of data collection, tables, or figures.
	
\end{appendix}

% ----------------------------------------------------------------------------
% Bibliography
% ----------------------------------------------------------------------------
\clearpage
\renewcommand\refname{Bibliography}
\addcontentsline{toc}{section}{Bibliography}
\bibliography{bibliography}
\bibliographystyle{plainnat}

% ----------------------------------------------------------------------------
% Statutory declaration
% ----------------------------------------------------------------------------
\clearpage
\makeThesisDeclaration

\end{document}

